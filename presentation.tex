\documentclass{beamer}
\usepackage[utf8]{inputenc}
\usepackage{xeCJK}
\usepackage{graphicx}
\usepackage{amsmath}

\setCJKmainfont{PingFang TC}

\title{哥布林介紹}
\author{作者名稱}
\date{\today}

\begin{document}

\begin{frame}
    \titlepage
\end{frame}

\begin{frame}{目錄}
    \tableofcontents
\end{frame}

\section{哥布林是什麼?}
\begin{frame}{哥布林是什麼?}
    \begin{itemize}
        \item 哥布林是一種出現在歐洲民間傳說中的小型妖精或怪物。
        \item 常見於奇幻文學、遊戲與電影中。
    \end{itemize}
\end{frame}

\section{哥布林的特徵}
\begin{frame}{哥布林的特徵}
    \begin{itemize}
        \item 身材矮小,皮膚多為綠色或灰色。
        \item 通常有尖耳朵、大鼻子、銳利的牙齒。
        \item 性格狡猾、貪婪,有時帶點幽默感。
    \end{itemize}
\end{frame}

\section{哥布林在文化中的形象}
\begin{frame}{哥布林在文化中的形象}
    \begin{itemize}
        \item 在《魔戒》、《哈利波特》等作品中都有出現。
        \item 遊戲如《魔獸世界》、《龍與地下城》也有各種哥布林角色。
    \end{itemize}
\end{frame}

\section{哥布林的數學趣談}
\begin{frame}{哥布林的數學趣談}
    假設有 5 個哥布林要分 20 枚金幣,每個哥布林至少要分到 2 枚,請問有多少種分法?
    \[
    \text{這是一個經典的整數分拆問題。}
    \]
\end{frame}

\end{document} 