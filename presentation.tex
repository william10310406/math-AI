\documentclass{beamer}
% 使用 XeLaTeX 編譯
\usepackage{fontspec}
\usepackage{xeCJK}
% Mac 系統預設繁體中文常用字體
\setCJKmainfont{PingFang TC}

\usetheme{Madrid}

\title{TWSIAM 2025 演講}
\author{}
\date{}

\begin{document}

\frame{\titlepage}

% 定義章節順序,以便目錄顯示完整
\section{神經網路認知模型導論}
\section{哥布林是什麼?}
\section{哥布林的特徵}
\section{哥布林在文化中的形象}
\section{哥布林的數學趣談}

\begin{frame}{目錄}
  \tableofcontents
\end{frame}

\begin{frame}{神經網路認知模型導論}
  \begin{itemize}
    \item Connectionist 模型:把大腦想像成一群互連的節點,用演算法模擬它如何處理信息。
    \item \textbf{McCulloch-Pitts}: 最原始的神經元模型,就像一個只能開/關的開關(0/1 輸入輸出)。
    \item \textbf{Perceptron}: 把多個開關串在一起,能學習簡單的分類(像判斷郵件是否為垃圾郵件)。
    \item \textbf{ADALINE}: 在 Perceptron 基礎上可微分,透過誤差最小化自動調整權重,更穩定的學習方式。
  \end{itemize}
\end{frame}

\begin{frame}{神經網路認知模型導論(續)}
  \begin{itemize}
    \item \textbf{Multilayer Perceptron}: 加入隱藏層和非線性函數,就能解決複雜問題,比如辨識手寫數字。
    \item \textbf{Convolutional NN}: 專門處理圖像,透過「卷積」找特徵、「池化」壓縮資訊,就像圖像打馬賽克前先找重點。
    \item \textbf{Recurrent NN}: 適合序列型資料(如語音、文字),能記憶前後文(Elman, LSTM)。
  \end{itemize}
\end{frame}

\begin{frame}{哥布林是什麼?}
  \begin{itemize}
    \item 哥布林是一種出現在歐洲民間傳說中的小型妖精或怪物。
    \item 常見於奇幻文學、遊戲與電影中。
  \end{itemize}
\end{frame}

\begin{frame}{哥布林的特徵}
    \begin{itemize}
        \item 身材矮小,皮膚多為綠色或灰色。
        \item 通常有尖耳朵、大鼻子、銳利的牙齒,性格狡猾、貪婪,有時帶點幽默感。
    \end{itemize}
\end{frame}

\begin{frame}{哥布林在文化中的形象}
  \begin{itemize}
    \item 在《魔戒》、《哈利波特》等作品中都有出現。
    \item 遊戲如《魔獸世界》、《龍與地下城》也有各種哥布林角色。
  \end{itemize}
\end{frame}

\begin{frame}{哥布林的數學趣談}
  \begin{itemize}
    \item 假設有 5 個哥布林要分 20 枚金幣,每個至少分到 2 枚,問分法數量。
  \end{itemize}
  \[
    \#\{(x_1,\dots,x_5)\in\mathbb{Z}^5\mid x_i\ge2,\sum_{i=1}^5 x_i=20\}
  \]
\end{frame}

\end{document}
